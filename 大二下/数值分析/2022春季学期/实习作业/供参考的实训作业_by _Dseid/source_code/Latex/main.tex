\PassOptionsToPackage{quiet}{fontspec}%去除莫名其妙的一个警告,这个得放documentclass前面


\documentclass{ctexart}%ctex article
\usepackage{silence}




\WarningFilter*{transparent}{Loading aborted, because pdfTeX is not running}%去除莫名其妙的又一个警告,这里前面是出问题包的名字,后面是报错命令的前面的一部分
\usepackage{graphicx}
\usepackage{geometry}%调整页面边距到整个页面的0.8倍
\geometry{a4paper,scale=0.75}
% 该宏包还可以设置页面的上下左右边距,例如:
% \geometry{a4paper,left=2cm,right=2cm,top=1cm,bottom=1cm}

\usepackage{listings}%用来插入代码块
\usepackage{xcolor}%辅助插入代码块
\definecolor{dkgreen}{RGB}{34,139,34}%上面那个包有了也不行 于是我定义这个颜色
\definecolor{mauve}{RGB}{224,176,255}%上面那个包有了也不行 于是我定义这个颜色

\lstset{ 
language=Octave,                % the language of the code
basicstyle=\footnotesize,           % the size of the fonts that are used for the code
numbers=left,                   % where to put the line-numbers
numberstyle=\tiny\color{gray},  % the style that is used for the line-numbers
stepnumber=2,                   % the step between two line-numbers. If it's 1, each line 
                                % will be numbered
numbersep=5pt,                  % how far the line-numbers are from the code
backgroundcolor=\color{white},      % choose the background color. You must add \usepackage{color}
showspaces=false,               % show spaces adding particular underscores
showstringspaces=false,         % underline spaces within strings
showtabs=false,                 % show tabs within strings adding particular underscores
frame=single,                   % adds a frame around the code
rulecolor=\color{black},        % if not set, the frame-color may be changed on line-breaks within not-black text (e.g. commens (green here))
tabsize=2,                      % sets default tabsize to 2 spaces
captionpos=b,                   % sets the caption-position to bottom
breaklines=true,                % sets automatic line breaking
breakatwhitespace=false,        % sets if automatic breaks should only happen at whitespace
title=\lstname,                 % show the filename of files included with \lstinputlisting;
                                % also try caption instead of title
keywordstyle=\color{blue},          % keyword style
commentstyle=\color{dkgreen},       % comment style
stringstyle=\color{mauve},         % string literal style
escapeinside={\%*}{*)},            % if you want to add LaTeX within your code
morekeywords={*,...}               % if you want to add more keywords to the set
}

\usepackage[inkscapelatex=false]{svg} % 需使用包svg
\usepackage{caption}
\usepackage{amsmath}%长公式对齐
\usepackage{amssymb}%如果不使用这个包/leqslant会冲突一直报错
\usepackage{float}%处理图片
\newtheorem{theorem}{定理}%处理定理编号
\usepackage{color}   %May be necessary if you want to color links
\usepackage{hyperref}%使得目录可以点击的宏包,颜色下面可以定义。
\hypersetup{
    colorlinks=true, %set true if you want colored links
    linktoc=all,     %set to all if you want both sections and subsections linked
    linkcolor=black,  %choose some color if you want links to stand out
}

\usepackage{fancyhdr} %通过使用这个好包重新定义了page number
\fancyhf{} % sets both header and footer to nothing
\renewcommand{\headrulewidth}{0pt}
% your new footer definitions here
\cfoot{\thepage}
\pagestyle{fancy}
% 自定义超链接
\usepackage{blindtext}
\usepackage{hyperref}
\graphicspath{ {images/} }
%toc means table of contents
\setcounter{tocdepth}{2}
\title { 数值分析实习作业 } 
\author { {\large \textit{Dseid}}\\ {\small 本实训作业满分,但仅供参考} }
\date { 2022年5月8日 }

\begin{document}
\maketitle

\tableofcontents
\thispagestyle{empty}
%上面这个语句是用来让这一页不要显示页眉页脚,也就不会显示page number了
\clearpage

\section{任务一}
\setcounter{page}{1}
\subsection{等距节点\textit{Lagrange}插值}
\subsubsection{算法描述}
拉格朗日插值法根据n+1个点构造插值基函数构造n次多项式对原函数拟合,本质上就是n次多项式插值。倘若选择的多项式次数过高,容易出现龙格现象。如\ref{lagrange}两侧所示出现了偏离真实函数值较大的情况。此处选择的插值点为:-5,4,...,4,5。多项式次数为10次
\subsubsection{图像展示}
\begin{figure}[htp]
    \centering
    \includesvg[width=\textwidth]{lagrange.svg}
    \caption{等距节点\textit{Lagrange}插值}
    \label{lagrange}
\end{figure}

\subsubsection{误差计算}
采用区间[‐5,5]上 101 个等距分布$\mathbf{z_j=-5+\frac{j}{10}},j=0,1,...,100$
处的误差绝对值的平均值来衡量。基于\textit{Matlab}计算得绝对误差为0.2858

\subsection{\textit{Chebyshev节点上的}\textit{Lagrange}插值}
\subsubsection{算法描述}
切比雪夫多项式 $T_{n}(x)$ 在区间 $[-1,1]$ 上有 $n$ 个零点
$$
x_{k}=\cos \frac{2 k-1}{2 n} \pi, \quad k=1,2, \cdots, n
$$
和 $n+1$ 个极值点 (包括端点)
$$
x_{k}=\cos \frac{k \pi}{n}, \quad k=0,1, \cdots, n .
$$
这两组点称为切比雪夫点, 它们在插值中有重要作用。\\
在所有首项系数为1的n次多项式集合$\tilde{H_n}$中
$$\left \| \tilde{T_n} \right \|_\infty = \min\limits_{P\in  \tilde{H}_n}   \left \| P(x) \right \|_\infty$$
设插值节点 $x_{0}, x_{1}, \cdots, x_{n}$ 为切比雪夫多项式 $T_{n+1}(x)$ 的零点, 被插函数 $f \in C^{n+1}[-1,1], L_{n}(x)$ 为相应的插值多项式, 则
$$
\max _{-1 \leqslant x \leqslant 1}\left|f(x)-L_{n}(x)\right| \leqslant \frac{1}{2^{n}(n+1) !}\left\|f^{(n+1)}(x)\right\|_{\infty} .
$$
对于一般区间 $[a, b]$ 上的插值只要利用变换 (2.14)式则可得到相应结果, 此时插值节点为
$$
x_{k}=\frac{b-a}{2} \cos \frac{2 k+1}{2(n+1)} \pi+\frac{a+b}{2}, \quad k=0,1, \cdots, n .
$$
由于高次插值出现龙格现象,一般 $L_{n}(x)$ 不收敛于 $f(x)$, 因此等距节点高次插值并不适用. 但若用切比雪夫多项式零点插值却可避免龙格现象, 可保证整个区间上收玫.
如图\ref{chebyshev}两侧拉格朗日插值多项式和原函数拟合的很好。

\subsubsection{图像展示}
\begin{figure}[H]
    \centering
    \includesvg[width=\textwidth]{chebyshev.svg}
    \caption{Chebyshev节点\textit{Lagrange}插值}
    \label{chebyshev}
\end{figure}

\subsubsection{误差计算}
采用区间[‐5,5]上 101 个等距分布$\mathbf{z_j=-5+\frac{j}{10}},j=0,1,...,100$
处的误差绝对值的平均值来衡量。基于\textit{Matlab}计算得绝对误差为0.0487,插值点数量和之前的等距节点数量相同,然而绝对误差却减少了十倍左右。

\subsection{分段线性插值}
\subsubsection{算法描述}
分段线性插值就是通过插值点用折线段连接起来逼近 $f(x)$. 设已知节点 $a=x_{0}<x_{1}$ $<\cdots<x_{n}=b$ 上的函数值 $f_{0}, f_{1}, \cdots, f_{n}$, 记 $h_{k}=x_{k+1}-x_{k}, h=\max\limits_{k} h_{k}$, 求一折线函数 $I_{h}(x)$ 满足:\\
(1) $I_{h}(x) \in C[a, b]$;\\
(2) $I_{h}\left(x_{k}\right)=f_{k}(k=0,1, \cdots, n)$;\\
(3) $I_{h}(x)$ 在每个小区间 $\left[x_{k}, x_{k+1}\right]$ 上是线性函数. 则称 $I_{h}(x)$ 为分段线性插值函数.如图\ref{linear}所示
% \newpage
\subsubsection{图像展示}
\begin{figure}[H]
    \centering
    \includesvg[width=\textwidth]{linear.svg}
    \caption{分段线性插值插值}
    \label{linear}
\end{figure}

\subsubsection{误差计算}
采用区间[‐5,5]上 101 个等距分布$\mathbf{z_j=-5+\frac{j}{10}},j=0,1,...,100$
处的误差绝对值的平均值来衡量。基于\textit{Matlab}计算得绝对误差为0.0148,插值点数量和之前的等距节点数量相同,然而绝对误差相比之前两种多项式插值方法都要更少。这说明在节点数量相同的情况下,相比于多项式插值,线性插值更适合用来对原函数进行拟合。




\subsection{三次样条插值}
\subsubsection{算法描述}

若函数 $S(x) \in C^{2}[a, b]$, 且在每个小区间 $\left[x_{j}, x_{j+1}\right]$ 上是三次多项式,其中 $a=$ $x_{0}<x_{1}<\cdots<x_{n}=b$ 是给定节点, 则称 $S(x)$ 是节点 $x_{0}, x_{1}, \cdots, x_{n}$ 上的三次样条函数. 若在节点 $x_{j}$ 上给定函数值 $y_{j}=f\left(x_{j}\right)(j=0,1, \cdots, n)$, 并成立
$$
S\left(x_{j}\right)=y_{j}, \quad j=0,1, \cdots, n,
$$
则称 $S(x)$ 为三次样条插值函数。
从定义知要求出 $S(x)$, 在每个小区间 $\left[x_{j}, x_{j+1}\right]$ 上要确定 4 个待定系数, 而共有 $n$ 个小 区间,故应确定 $4 n$ 个参数. 根据 $S(x)$ 在 $[a, b]$ 上二阶导数连续,在节点 $x_{j}(j=1,2, \cdots, n-1)$ 处应满足连续性条件
$$
S\left(x_{j}-0\right)=S\left(x_{j}+0\right)$$ $$\quad S^{\prime}\left(x_{j}-0\right)=S^{\prime}\left(x_{j}+0\right)$$ $$\quad S^{\prime \prime}\left(x_{j}-0\right)=S^{\prime \prime}\left(x_{j}+0\right) 
$$ 这里共有 $3 n-3$ 个条件,再加上 $S(x)$ 满足插值条件, 共有 $4 n-2$ 个条件,因此还需要 加上 2 个条件才能确定 $S(x)$. 通常可在区间 $[\cdot a, b]$ 的端点 $a=x_{0}, b=x_{n}$ 上各加一个条件 (称 为边界条件), 可根据实际问题的要求给定. 本次插值使用第一类边界条件:已知两端的一阶导数值\\
即
$$
S^{\prime}\left(x_{0}\right)=f_{0}^{\prime}, \quad S^{\prime}\left(x_{n}\right)=f_{n}^{\prime} .
$$
相比分段线性插值在区间的端点不连续,三次样条插值可以保证二阶导数值连续,图\ref{spline}中直观的诠释了三次样条插值这一突出特性。

\subsubsection{图像展示}
\begin{figure}[H]
    \centering
    \includesvg[width=\textwidth]{spline.svg}
    \caption{三次样条插值}
    \label{spline}
\end{figure}

\subsubsection{误差计算}
设 $f(x) \in C^{4}[a, b], S(x)$ 为满足第一种或第二种边界条件的三次样条函数, 令 $h=\max _{0 \leqslant i \leqslant n-1} h_{i}, h_{i}=x_{i+1}-x_{i}(i=0,1, \cdots, n-1)$, 则有估计式: 
$$\max _{a \leqslant x \leqslant b}\left|f^{(k)}(x)-S^{(k)}(x)\right| \leqslant C_{k} \max _{a \leqslant x \leqslant b}\left|f^{(4)}(x)\right| h^{4-k}, \quad k=0,1,2$$
\\其中 $C_{0}=\frac{5}{384}, C_{1}=\frac{1}{24}, C_{2}=\frac{3}{8}$.

采用区间[‐5,5]上 101 个等距分布$\mathbf{z_j=-5+\frac{j}{10}},j=0,1,...,100$
处的误差绝对值的平均值来衡量。基于\textit{Matlab}计算得绝对误差为0.0040,插值点数量和之前的等距节点数量相同,然而绝对误差相比之前三种方法都要更少,从图像上也可以直观的看出拟合的优度。这说明在节点数量相同的情况下,相比于多项式插值、线性插值,三次样条插值很好的保留了原函数的光滑特性,达到了更好的拟合优度。




\subsection{\textit{Legendre}多项式最佳平方逼近}
\subsubsection{算法描述}
如果

$$
\begin{aligned}
\left\|f(x)-P^{*}(x)\right\|_{2}^{2} &=\min _{P \in H_{n}}\|f(x)-P(x)\|_{2}^{2} \\
&=\min _{P \in H_{n}} \int_{a}^{b}[f(x)-P(x)]^{2} \mathrm{~d} x,
\end{aligned}
$$
则称 $P^{*}(x)$ 为 $f(x)$ 在 $[a, b]$ 上的最佳平方逼近多项式。
用 $\left\{1, x, \cdots, x^{n}\right\}$ 作基, 求最佳平方逼近多项式, 当 $n$ 较大时, 系数矩阵 是高度病态的, 因此直接求解法方程是相当困难的, 通常是采用正交多项式作基,如\textit{Legendre}多项式作为基底求最佳平方逼近多项式。\\
如果 $f(x) \in C[a, b]$, 求 $[a, b]$ 上的最佳平方逼近多项式, 做变换
$$
x=\frac{b-a}{2} t+\frac{b+a}{2} \quad(-1 \leqslant t \leqslant 1),
$$
于是 $F(t)=f\left(\frac{b-a}{2} t+\frac{b+a}{2}\right)$ 在 $[-1,1]$ 上可用勒让德多项式做最佳平方逼近多项式 $S_{n}^{*}(t)$, 从而得到区间 $[a, b]$ 上的最佳平方逼近多项式 $$S_{n}^{*}\left(\frac{1}{b-a}(2 x-a-b)\right)$$
由于勒让德多项式 $\left\{\mathrm{P}_{k}(x)\right\}$ 是在区间 $[-1,1]$ 上由 $\left\{1, x, \cdots, x^{k}, \cdots\right\}$ 正交化得到的, 因此 利用函数的勒让德展开部分和得到最佳平方逼近多项式与由$$
S(x)=a_{0}+a_{1} x+\cdots+a_{n} x^{n}
$$
直接通过解法方程得到 $H_{n}$ 中的最佳平方逼近多项式是一致的,只是当 $n$ 较大时法方程出 现病态,计算误差较大, 不能使用,而用勒让德展开不用解线性方程组,不存在病态问题,计算公式比较方便, 因此通常都用这种方法求最佳平方逼近多项式。如图\ref{legendre}所示,采用四阶勒让德多项式进行逼近
\subsubsection{图像展示}
\begin{figure}[H]
    \centering
    \includesvg[width=\textwidth]{legendre.svg}
    \caption{\textit{Legendre}多项式拟合}
    \label{legendre}
\end{figure}

\subsubsection{误差计算}
定理:设 $f(x) \in C^{2}[-1,1], S_{n}^{*}(x)$是用勒让德多项式拟合原函数得到的多项式, 则对任意 $x \in[-1,1]$ 和 $\forall \varepsilon>0$, 当 $n$ 充分大时有
$$
\left|f(x)-S_{n}^{*}(x)\right| \leqslant \frac{\varepsilon}{\sqrt{n}} .
$$
采用区间[‐5,5]上 101 个等距分布$\mathbf{z_j=-5+\frac{j}{10}},j=0,1,...,100$
处的误差绝对值的平均值来衡量。基于\textit{Matlab}计算得绝对误差为0.1082,由于选择的勒让德的多项式阶数不高,拟合绝对误差相比前面几种方法稍有逊色。



\subsection{最小二乘法拟合多项式}
\subsubsection{算法描述}
在函数的最佳平方逼近中 $f(x) \in C[a, b]$, 如果 $f(x)$ 只在一组离散点集 $\left\{x_{i}, i=0\right.$, $1, \cdots, m\}$ 上给出, 这就是科学实验中经常见到的实验数据 $\left\{\left(x_{i}, y_{i}\right), i=0,1, \cdots, m\right\}$ 的曲线拟 合, 这里 $y_{i}=f\left(x_{i}\right)(i=0,1, \cdots, m)$, 要求一个函数 $y=S^{*}(x)$ 与所给数据 $\left\{\left(x_{i}, y_{i}\right), i=0\right.$, $1, \cdots, m\}$ 拟合, 若记误差 $\delta_{i}=S^{*}\left(x_{i}\right)-y_{i}(i=0,1, \cdots, m), \delta=\left(\delta_{0}, \delta_{1}, \cdots, \delta_{m}\right)^{\mathrm{T}}$, 设 $\varphi_{0}(x)$, $\varphi_{1}(x), \cdots, \varphi_{n}(x)$ 是 $C[a, b]$ 上线性无关函数族, 在 $$\varphi=\operatorname{span}\left\{\varphi_{0}(x),  \varphi_{1}(x), \cdots, \varphi_{n}(x)\right\}$$ 中找一 函数 $S^{*}(x)$, 使误差平方和
$$
\|\delta\|_{2}^{2}=\sum_{i=0}^{m} \delta_{i}^{2}=\sum_{i=0}^{m}\left[S^{*}\left(x_{i}\right)-y_{i}\right]^{2}=\min _{S(x) \in \varphi} \sum_{i=0}^{m}\left[S\left(x_{i}\right)-y_{i}\right]^{2},
$$
这里
$$
S(x)=a_{0} \varphi_{0}(x)+a_{1} \varphi_{1}(x)+\cdots+a_{n} \varphi_{n}(x) \quad(n<m) .
$$
这就是一般的最小二乘逼近, 用几何语言说, 就称为曲线拟合的最小二乘法。
采用四次多项式最小二乘法,在-5到5的间距为1的节点上拟合原函数,如图\ref{least_squares}所示
\subsubsection{图像展示}
\begin{figure}[H]
    \centering
    \includesvg[width=\textwidth]{least_squares.svg}
    \caption{最小二乘法拟合多项式}
    \label{least_squares}
\end{figure}

\subsubsection{误差计算}
定理:设 $f(x) \in C^{2}[-1,1], S_{n}^{*}(x)$是用勒让德多项式拟合原函数得到的多项式, 则对任意 $x \in[-1,1]$ 和 $\forall \varepsilon>0$, 当 $n$ 充分大时有
$$
\left|f(x)-S_{n}^{*}(x)\right| \leqslant \frac{\varepsilon}{\sqrt{n}} .
$$
这意味着$
\text { 如果 } f(x) \text { 满足光浯性条件还可得到 } S_{n}^{*}(x) \text { 一致收敛于 } f(x) \text { 的结论. }
$

采用区间[‐5,5]上 101 个等距分布$\mathbf{z_j=-5+\frac{j}{10}},j=0,1,...,100$
处的误差绝对值的平均值来衡量。基于\textit{Matlab}计算得绝对误差为0.1115,与四次勒让德的多项式逼近类似,由于选择的阶数不高,拟合绝对误差大体上与四次勒让德的多项式逼近相同。


\newpage
\section{任务二}

\subsection{算法描述及误差阶分析}
\hypertarget{wucha}{}
\subsubsection{复合梯形公式}
将区间 $[a, b]$ 划分为 $n$ 等份, 分点 $x_{k}=a+k h, h=\frac{b-a}{n}, k=0,1, \cdots, n$, 在每个子区间 $\left[x_{k}, x_{k+1}\right](k=0,1, \cdots, n-1)$ 上采用梯形公式, 则得
$$
I=\int_{a}^{b} f(x) \mathrm{d} x=\sum_{k=0}^{n-1} \int_{x_{k}}^{x_{k+1}} f(x) \mathrm{d} x=\frac{h}{2} \sum_{k=0}^{n-1}\left[f\left(x_{k}\right)+f\left(x_{k+1}\right)\right]+R_{n}(f) .
$$
记
$$
T_{n}=\frac{h}{2} \sum_{k=0}^{n-1}\left[f\left(x_{k}\right)+f\left(x_{k+1}\right)\right]=\frac{h}{2}\left[f(a)+2 \sum_{k=1}^{n-1} f\left(x_{k}\right)+f(b)\right],
$$
称为复合梯形公式, 其余项
$$
R_{n}(f)=I-T_{n}=\sum_{k=0}^{n-1}\left[-\frac{h^{3}}{12} f^{\prime \prime}\left(\eta_{k}\right)\right], \quad \eta_{k} \in\left(x_{k}, x_{k+1}\right) .
$$
由于 $f(x) \in C^{2}[a, b]$, 且
$$
\min _{0 \leqslant k \leqslant n-1} f^{\prime \prime}\left(\eta_{k}\right) \leqslant \frac{1}{n} \sum_{k=0}^{n-1} f^{\prime \prime}\left(\eta_{k}\right) \leqslant \max _{0 \leqslant k \leqslant n-1} f^{\prime \prime}\left(\eta_{k}\right),
$$
所以 $\exists \eta \in(a, b)$ 使$$
f^{\prime \prime}(\eta)=\frac{1}{n} \sum_{k=0}^{n-1} f^{\prime \prime}\left(\eta_{k}\right)
$$
于是复合梯形公式的余项为
$$
R_{n}(f)=-\frac{b-a}{12} h^{2} f^{\prime \prime}(\eta) .
$$
可以看出误差是 $h^{2}$ 阶。
\subsubsection{复合辛普森公式}

将区间 $[a, b]$ 分为 $n$ 等份, 在每个子区间 $\left[x_{k}, x_{k+1}\right]$ 上采用辛普森公式 (2.3), 若记 $x_{k+1 / 2}=x_{k}+\frac{1}{2} h$, 则得
$$
\begin{aligned}
I &=\int_{a}^{b} f(x) \mathrm{d} x=\sum_{k=0}^{n-1} \int_{x_{k}}^{x_{k+1}} f(x) \mathrm{d} x \\
&=\frac{h}{6} \sum_{k=0}^{n-1}\left[f\left(x_{k}\right)+4 f\left(x_{k+1 / 2}\right)+f\left(x_{k+1}\right)\right]+R_{n}(f) .
\end{aligned}
$$
记
$$
\begin{aligned}
S_{n} &=\frac{h}{6} \sum_{k=0}^{n-1}\left[f\left(x_{k}\right)+4 f\left(x_{k+1 / 2}\right)+f\left(x_{k+1}\right)\right] \\
&=\frac{h}{6}\left[f(a)+4 \sum_{k=0}^{n-1} f\left(x_{k+1 / 2}\right)+2 \sum_{k=1}^{n-1} f\left(x_{k}\right)+f(b)\right],
\end{aligned}
$$
称为复合辛普森求积公式, 其余项由 (2.5) 式得
$$
R_{n}(f)=I-S_{n}=-\frac{h}{180}\left(\frac{h}{2}\right)^{4} \sum_{k=0}^{n-1} f^{(4)}\left(\eta_{k}\right), \quad \eta_{k} \in\left(x_{k}, x_{k+1}\right) .
$$
于是当 $f(x) \in C^{4}[a, b]$ 时, 与复合梯形公式相似有$$
R_{n}(f)=I-S_{n}=-\frac{b-a}{180}\left(\frac{h}{2}\right)^{4} f^{(4)}(\eta), \quad \eta \in(a, b) .
$$
由 (3.6) 式看出, 误差阶为 $h^{4}$。

\subsubsection{龙贝格积分算法}
从梯形公式出发, 将区间 $[a, b]$ 逐次二分可提高求积公式精度, 当 $[a, b]$ 分为 $n$ 等份 时, 有
$$
I-T_{n}=-\frac{b-a}{12} h^{2} f^{\prime \prime}(\eta), \quad \eta \in[a, b], \quad h=\frac{b-a}{n} .
$$
若记 $T_{n}=T(h)$, 当区间 $[a, b]$ 分为 $2 n$ 等份时, 则有 $T_{2 n}=T\left(\frac{h}{2}\right)$, 并且有
$$
T(h)=I+\frac{b-a}{12} h^{2} f^{\prime \prime}(\eta), \quad \lim _{h \rightarrow 0} T(h)=T(0)=I,
$$
可以证明梯形公式的余项可展成级数形式, 即有下面定理.\\
\begin{theorem}
设 $f(x) \in C^{\infty}[a, b]$, 则有
$$
T(h)=I+\alpha_{1} h^{2}+\alpha_{2} h^{4}+\cdots+\alpha_{l} h^{2 l}+\cdots,
$$
其中系数 $\alpha_{l}(l=1,2, \cdots)$ 与 $h$ 无关.
此定理可利用 $f(x)$ 的泰勒展开推导得到, 此处从略。
\label{theorem1}
\end{theorem} 
 定理\ref{theorem1}表明 $T(h) \approx I$ 是 $O\left(h^{2}\right)$ 阶, 在 (4.2)式中, 若用 $h / 2$ 代替 $h$, 有
$$
T\left(\frac{h}{2}\right)=I+\alpha_{1} \frac{h^{2}}{4}+\alpha_{2} \frac{h^{4}}{16}+\cdots+\alpha_{l}\left(\frac{h}{2}\right)^{2 l}+\cdots .
$$
 则有
$$
S(h)=\frac{4 T(h / 2)-T(h)}{3}=I+\beta_{1} h^{4}+\beta_{2} h^{6}+\cdots,
$$
这里 $\beta_{1}, \beta_{2}, \cdots$ 是与 $h$ 无关的系数. 用 $S(h)$ 近似积分值 $I$, 其误差阶为 $O\left(h^{4}\right)$, 这比复合梯形 公式的误差阶 $O\left(h^{2}\right)$ 提高了, 容易看到 $S(h)=S_{n}$, 即将 $[a, b]$ 分为 $n$ 等份得到的复合辛普森公式\footnote{稍后会验证区间等分10份的复合辛普森公式与经过一次龙贝格加速的区间等分10份的复合梯形公式的结果完全一致}. 这种将计算 $I$ 的近似值的误差阶由 $O\left(h^{2}\right)$ 提高到 $O\left(h^{4}\right)$ 的方法称为外推算法。
\\设以 $T_{0}^{(k)}$ 表示二分 $k$ 次后求得的梯形值, 且以 $T_{m}^{(k)}$ 表示序列 $\left\{T_{0}^{(k)}\right\}$ 的 $m$ 次加速值, 则 依递推公式 (4.9) 可得
$$
T_{m}^{(k)}=\frac{4^{m}}{4^{m}-1} T_{m-1}^{(k+1)}-\frac{1}{4^{m}-1} T_{m-1}^{(k)}, \quad k=1,2, \cdots .
$$
公式 (4.11) 也称为龙贝格求积算法。

\subsubsection{高斯求积公式}
形如下式的机械求积公式
$$
\int_{a}^{b} f(x) \mathrm{d} x \approx \sum_{k=0}^{n} A_{k} f\left(x_{k}\right)
$$
含有 $2 n+2$ 个待定参数 $x_{k}, A_{k}(k=0,1, \cdots, n)$. 当 $x_{k}$ 为等距节点时得到的插值求积公式其 代数精度至少为 $n$ 次, 如果适当选取 $x_{k}(k=0,1, \cdots, n)$, 有可能使求积公式具有 $2 n+1$ 次代 数精度。
如果求积公式具有 $2 n+1$ 次代数精度, 则称其节点 $x_{k}(k=0,1, \cdots, n)$ 为 高斯点, 相应公式 称为高斯型求积公式.
在高斯求积公式中, 若取权函数 $\rho(x)=1$, 区间为 $[-1,1]$, 则得公式
$$
\int_{-1}^{1} f(x) \mathrm{d} x \approx \sum_{k=0}^{n} A_{k} f\left(x_{k}\right) .
$$
我们知道勒让德多项式是区间 $[-1,1]$ 上的正交多项式, 因此, 勒让德多项式 $\mathrm{P}_{n+1}(x)$ 的零点就是求积公式的高斯点. 此时称为高斯-勒让德求积公式。
\begin{table}[H]
$$
\begin{array}{c|c|c}
\hline n & x_{k}  & A_{k} \\
\hline 0 & 0.0000000 & 2.0000000 \\
\hline 1 & \pm 0.5773503 & 1.0000000 \\
\hline 2 & \pm 0.7745967 & 0.5555556 \\
&0.0000000 & 0.8888889 \\
% \qquad%这个命令可以将两个表横着连在一起
\hline 3 & \pm 0.8611363 & 0.3478548 \\
&\pm 0.3399810 & 0.6521452 \\
\hline 4 & \pm 0.9061798 & 0.2369269 \\
& \pm 0.5384693 & 0.4786287 \\
& 0.0000000 & 0.5688889 \\
\hline 5 & \pm 0.9324695 & 0.1713245 \\
&\pm 0.6612094 & 0.3607616 \\
&\pm 0.2386192 & 0.4679139 \\
\hline
\end{array}
$$
\caption{高斯-勒让德求积公式的节点和系数}
\end{table}

复合高斯公式将区间 $[a, b]$ 分为 $n$ 等份, 在每个子区间 $\left[x_{k}, x_{k+1}\right]$ 上采用高斯公式。\\
\textbf{下面讨论 Gauss 求积公式的误差}
\begin{theorem}
定理 设$f \in C^{2 n+2}[a, b]$, 那么 Gauss 求积公式 (4.8) 的误差表达式为
$$
E_{n}(f)=\frac{1}{(2 n+2) !} f^{(2 n+2)}(\eta) \int_{a}^{b} \rho(x)\left[\omega_{n+1}(x)\right]^{2} \mathrm{~d} x,
$$
其中 $\eta \in[a, b], \omega_{n+1}(x)=\left(x-x_{0}\right)\left(x-x_{1}\right) \cdots\left(x-x_{n}\right)$

%\footnote{在这里写脚注}这样可以插入脚注
\label{theorem2}
\end{theorem} 
设$\left(b-a\right)$=${h}$,则$$
\omega_{n+1}(x)=\left(x-x_{0}\right)\left(x-x_{1}\right) \cdots\left(x-x_{n}\right)
$$可以近似的视作为$h^{m}$阶,则易得复合高斯公式的误差为 $h^{2m}$阶,其中${m}$为高斯点的个数。

\subsection{误差计算}
采用区间[‐5,5]上 101 个等距分布$\mathbf{z_j=-5+\frac{j}{10}},j=0,1,...,100$
处的误差绝对值的平均值来衡量。基于\textit{Matlab}计算得各种方法计算积分结果与误差如下表所示:

\begin{table}[H]
$$
\begin{array}{c|c|c}
\hline method & value  & bias \\
\hline trapz & 2.2132 & 1.1\times{10}^{-3} \\
\hline simpsons &  2.2143 & 4.3816\times{10}^{-6} \\
\hline Romberg1 & 2.2143 & 4.3816\times{10}^{-6} \\
\hline Romberg2 & 2.2142 & 5.4711\times{10}^{-5} \\
\hline gauss\_2points & 2.2143 & 2.9372\times{10}^{-6} \\
\hline gauss\_4points & 2.2143 & 2.7243\times{10}^{-6} \\
\hline
\end{array}
$$
\caption{不同方法计算积分结果}
\label{outcome}%这行似乎得放在caption下面 否则无法正常引用
\end{table}

\subsection{误差比较与评价}
由表\ref{outcome}可知,复合四点高斯公式的误差最小,其次是复合两点高斯公式。而在复合梯形公式基础上改进的复合辛普森公式、龙贝格积分法的误差均相比于复合梯形公式减少了100倍以上。值得注意的是复合梯形公式经过一次龙贝格加速后所得到的结果,和辛普森公式所得到的完全一致。这与理论推导相照应。\\
此外,本次实验中不难发现,龙贝格二次加速得到的结果竟然比龙贝格一次加速的结果误差更大,经过和室友、同学认真的讨论,结论是:这仅仅是偶然现象,虽然龙贝格二次加速的误差阶数绝对会更小,却不意味着绝对误差一定会小于一次龙贝格加速的误差。此外,经过尝试三次,四次龙贝格加速之后,误差总体上仍然显示为下降趋势,这意味着算法本身问题不大。本次实验中遇到的情况的属于正常现象。




\section{选做题}
\subsection{积分公式误差阶理论分析}
根据2.1节(\hyperlink{wucha}{\textbf{见此处}})算法描述中的详细分析,各种积分公式的误差阶如表\ref{order}所示:
\begin{table}[H]
$$
\begin{array}{c|c|c}
\hline method & bias_order \\
\hline trapz & h^{2} \\
\hline simpsons & h^{4}  \\
\hline gauss\_3points & h^{6} \\%\_才是下划线 不然会变成下标
\hline
\end{array}
$$
\caption{不同积分方法误差阶数}
\label{order}%这行似乎得放在caption下面 否则无法正常引用
\end{table}
\subsection{数值验证误差阶}
% (b) 取 $h=\frac{4}{n}, n=2^{k}, k=2,3,4,5,6$, 数值验证上述误差收敛阶 $O\left(h^{\alpha}\right)$ 

根据\textit{Matlab}计算,在k取不同的值时,通过$$\frac{\log \left(e_{k}\right)-\log \left(e_{k-1}\right)}{\log \left(h_{k}\right)-\log \left(h_{k-1}\right)}$$ 验证收敛阶 $O\left(h^{\alpha}\right)$ 中的阶数 $\alpha$  ,(其中 $h_{k}, e_{k}$ 分别为 $k=2,3,4,5,6$ 时 对应的区间长度和积分误差)
所得到的阶数如表\ref{val_order}所示:
\begin{table}[H]
$$
\begin{array}{c|c|c|c|c}
\hline k &3&4&5&6 \\
\hline trapz\_order &1.1141 &1.9888&1.9984&1.9996\\
\hline simpsons & 4.3245 &7.8773 & 4.7853&3.9970 \\
\hline gauss\_3points & 9.01449&12.52915&5.99745&5.9939 \\

\hline
\end{array}
$$
\caption{数值验证误差阶数}
\label{val_order}%这行似乎得放在caption下面 否则无法正常引用
\end{table}

通过表中数据可以明显看出,随着k值不断增大,不同积分公式的k值趋向于预估的值,说明理论推导正确。


\clearpage
\section{代码}
\subsection{任务一}
由于题目要求发生变化(不允许调用库函数),原始代码调用了库函数,现已经重新对所有调用库函数的地方做出了补充,补充了调用自己编的函数的方法,原调用库函数的方法予以保留。    \begin{flushright}---2022/05/18 
\end{flushright}       
\subsubsection{主函数部分}
\begin{lstlisting}
% 下面这里是lagrange插值插值
clc
%任务1.1
% 需要插值的点
x=linspace(-5,5,11);   
y=1./(1+x.^2);
%根据插值点构造插值多项式
xx=-5:0.01:5;
yy = lagrange(xx,x,y);
% 与真实函数图像做对比,f是真实函数图像。
f=1./(1+xx.^2);
plot(x,y,'o',xx,yy,xx,f,'--');
legend('插值点','lagrange插值多项式','被插值函数');
saveas(gcf,'lagrange.svg');
% 1.画出每个插值点,形式为圆圈
% 2.画出插值函数生成的图
% 3.画出真实函数图像以做对比

%误差计算
xxx=linspace(-5,5,101);
yyy = lagrange(xxx,x,y);
real_values=1./(1+xxx.^2);
abs(yyy-real_values);
bias=mean(abs(yyy-real_values))

clc
%任务1.2
% 需要插值的点
% find 11 Chebyshev nodes and mark them on the plot
n = 11;
k = 1:11; % iterator
xc = cos((2*k-1)/2/n*pi); % Chebyshev nodes
x=5.*xc;   
y=1./(1+x.^2);
%根据插值点构造插值多项式
xx=-5:0.01:5;
yy = lagrange(xx,x,y);
% 与真实函数图像做对比,f是真实函数图像。
f=1./(1+xx.^2);
plot(x,y,'o',xx,yy,xx,f,'--');
legend('插值点','chebyshev插值多项式','被插值函数');
saveas(gcf,'chebyshev.svg');
% 1.画出每个插值点,形式为圆圈
% 2.画出插值函数生成的图
% 3.画出真实函数图像以做对比

%误差计算
xxx=linspace(-5,5,101);
yyy = lagrange(xxx,x,y);
real_values=1./(1+xxx.^2);
abs(yyy-real_values);
bias=mean(abs(yyy-real_values))



clc
%任务1.3 (1)分段线性插值不调库法
syms x
y = 1/(1+x^2)
x0 = -5:5
y0 = 1./(1+x0.^2)
error = 0
for i = 1:10
    a=y0(i)*(x-x0(i+1))/(x0(i)-x0(i+1))+y0(i+1)*(x-x0(i))/(x0(i+1)-x0(i));
    for j = -6+i:0.1:-5+i
        error = error + abs(subs(a,x,j)-subs(y,x,j));
    end
    fplot(a,[-6+i,-5+i])
    hold on
end
a= eval(vpa(error,5))
fplot(y,[-5,5])
saveas(gcf,'another_linear.svg');
hold off



clc
%任务1.3 (2)分段线性插值调库法
% 需要插值的点
x=linspace(-5,5,11);   
y=1./(1+x.^2);
%根据插值点构造插值多项式
xx=-5:0.01:5;
yy = interp1(x,y,xx);
% 与真实函数图像做对比,f是真实函数图像。
f=1./(1+xx.^2);
plot(x,y,'o',xx,yy,xx,f,'--')
legend('插值点','线性插值多项式','被插值函数')
saveas(gcf,'linear.svg');
% 1.画出每个插值点,形式为圆圈
% 2.画出插值函数生成的图
% 3.画出真实函数图像以做对比

%误差计算
xxx=linspace(-5,5,101);
yyy = interp1(x,y,xxx);
real_values=1./(1+xxx.^2);
abs(yyy-real_values);
bias=mean(abs(yyy-real_values))






clc
%任务1.4 (1)三次样条插值调库法
% 需要插值的点
x=linspace(-5,5,11);   
y=1./(1+x.^2);

syms m
f = 1/(1+m^2);
diff_f = diff(f);
y1 = subs(diff_f,m,-5);
y2 = subs(diff_f,m,5);
y1=eval(y1);
y2=eval(y2);
y_spline=[y1 y y2];
%根据两端点的一阶导数值(clamped方法)
%即第一类边界条件
%根据插值点构造插值多项式
cs = spline(x,y_spline);

%计算三次样条插值各个密集点的值
xx=-5:0.01:5;
yy = ppval(cs,xx);
% 与真实函数图像做对比,f是真实函数图像。
f=1./(1+xx.^2);
plot(x,y,'o',xx,yy,xx,f,'--')
legend('插值点','三次样条插值多项式','被插值函数')
saveas(gcf,'spline.svg');
% 1.画出每个插值点,形式为圆圈
% 2.画出插值函数生成的图
% 3.画出真实函数图像以做对比

%误差计算
xxx=linspace(-5,5,101);
yyy = ppval(cs,xxx);
real_values=1./(1+xxx.^2);
abs(yyy-real_values);
bias=mean(abs(yyy-real_values))



clc
%任务1.4 (2)三次样条插值自编法
% 需要插值的点
x=linspace(-5,5,11);   
y=1./(1+x.^2);

syms m
f = 1/(1+m^2);
diff_f = diff(f);
y1 = subs(diff_f,m,-5);
y2 = subs(diff_f,m,5);
y1=eval(y1);
y2=eval(y2);
%根据两端点的一阶导数值(clamped方法)
%即第一类边界条件
%根据插值点构造插值多项式

%计算三次样条插值各个密集点的值
xx=-5:0.01:5;
yy = splineA(x,y,xx,[1,2],[y1,y2]);
% 与真实函数图像做对比,f是真实函数图像。
f=1./(1+xx.^2);
plot(x,y,'o',xx,yy,xx,f,'--')
legend('插值点','三次样条插值多项式','被插值函数')
saveas(gcf,'another_spline.svg');
% 1.画出每个插值点,形式为圆圈
% 2.画出插值函数生成的图
% 3.画出真实函数图像以做对比

%误差计算
xxx=linspace(-5,5,101);
yyy = ppval(cs,xxx);
real_values=1./(1+xxx.^2);
abs(yyy-real_values);
bias=mean(abs(yyy-real_values))

clc
%任务1.5 最佳平方逼近多项式
% 需要插值的点
x=linspace(-5,5,11);   
y=1./(1+x.^2);

syms m
legend_f=0;
f=1/(1+25*m^2);
for i = 0:4
    s=legendreP(i,m)*f;
    legend_f = legend_f+legendreP(i,m)*eval(int(s,m,-1,1))*(2*i+1)/2;
end
%计算勒让德多项式插值各个密集点的值
legend_f=subs(legend_f,m,m/5);
xx=-5:0.01:5;
yy = eval(subs(legend_f,m,xx));
% 与真实函数图像做对比,f是真实函数图像。
f=1./(1+xx.^2);
plot(x,y,'o',xx,yy,xx,f,'--')
legend('插值点','勒让德插值多项式','被插值函数')
saveas(gcf,'legendre.svg');
% 1.画出每个插值点,形式为圆圈
% 2.画出插值函数生成的图
% 3.画出真实函数图像以做对比

%误差计算
xxx=linspace(-5,5,101);
yyy = eval(subs(legend_f,m,xxx));
real_values=1./(1+xxx.^2);
abs(yyy-real_values);
bias=mean(abs(yyy-real_values))


clc 
%%%任务1.6 最小二乘拟合多项式
%% 需要插值的点
x_points=linspace(-5,5,11);   
y=1./(1+x_points.^2);
%根据插值点构造插值多项
syms x;
f = 1./(1+x.^2);
p = least_squares(f,x_points);
%此处由于不允许调库,所以改成手写
%如果调库,可以使用polyfit
xx=-5:0.01:5;
p4=subs(p,x,xx);
xxx=linspace(-5,5,101);
poly_vals=eval(subs(p,x,xxx));
%根据插值多项式生成一系列密集点用来画图

%与真实函数图像做对比,f是真实函数图像。
f=1./(1+xx.^2);
plot(x_points,y,'o',xx,p4,xx,f,'--');
legend('插值点','最小二乘拟合多项式','被插值函数');
saveas(gcf,'least_squares.svg');
%1.画出每个插值点,形式为圆圈
%2.画出插值函数生成的图
%3.画出真实函数图像以做对比

% 误差计算

real_values=1./(1+xxx.^2);
abs(poly_vals-real_values);
bias=mean(abs(poly_vals-real_values));


\end{lstlisting}
\subsubsection{自定义函数部分}
\begin{lstlisting}
function y = splineA(xd,yd,x,Ends,Ders)
n=length(xd);

h=diff(xd); h1=h(1); hn=h(n-1);
h=[h1 h1 h1 h hn hn hn];
Xd=[xd(1)-3*h1 xd(1)-2*h1 xd(1)-h1 xd xd(n)+hn xd(n)+2*hn xd(n)+3*hn];
[B,G]=coeffs(h);
if Ends(1)==2
    alpha0=G(4,1)/G(2,1);
    beta0=G(6,1)/G(2,1);
    gamma0=Ders(1)/G(2,1);
elseif  Ends(1)==1
    alpha0=G(3,1)/G(1,1);
    beta0=G(5,1)/G(1,1);
    gamma0=Ders(1)/G(1,1);
end
B0=Bi(xd(1),0,B,Xd);
Aa(1)=2/3-B0*alpha0;
Ac(1)=Bi(xd(1),2,B,Xd)-B0*beta0;
Ab(1)=0;
yd(1)=yd(1)-B0*gamma0;
for ii=2:n-1
    Aa(ii)=2/3;
    Ac(ii)=Bi(xd(ii),ii+1,B,Xd);
    Ab(ii)=Bi(xd(ii),ii-1,B,Xd);
end
if Ends(2)==2
    alphan=G(2,2)/G(6,2);
    betan=G(4,2)/G(6,2);
    gamman=Ders(1)/G(6,2);
elseif  Ends(2)==1
    alphan=G(1,2)/G(5,2);
    betan=G(3,2)/G(5,2);
    gamman=Ders(2)/G(5,2);
end
Bn1=Bi(xd(n),n+1,B,Xd);
Aa(n)=2/3-Bn1*betan;
Ac(n)=0;
Ab(n)=Bi(xd(n),n-1,B,Xd)-Bn1*alphan;
yd(n)=yd(n)-gamman*Bn1;
w=tri(Aa,Ab,Ac,yd);
a0=-alpha0*w(1)-beta0*w(2)+gamma0;
anp1=-alphan*w(n-1)-betan*w(n)+gamman;
w=[a0 w anp1];
nx=length(x);
y=zeros(1,nx);
for ix=1:nx
    sum=0;
    for k=1:n+2
        sum=sum+w(k)*Bi(x(ix),k-1,B,Xd);
    end
    y(ix)=sum;
end

% local functions used by algorithm
function y = tri( a, b, c, f )
N = length(f);
v = zeros(1,N);
y = v;
w = a(1);
y(1) = f(1)/w;
for i=2:N
    v(i-1) = c(i-1)/w;
    w = a(i) - b(i)*v(i-1);
    y(i) = ( f(i) - b(i)*y(i-1) )/w;
end
for j=N-1:-1:1
    y(j) = y(j) - v(j)*y(j+1);
end

function [B,G] =coeffs(h)
N=length(h);
B=zeros(N-3,16);
G=zeros(6,2);
for j=1:N-3
    hm2=h(j); hm1=h(j+1); h1=h(j+2); h2=h(j+3);
    d2=solver(hm2,hm1,h1,h2);
    B(j,4)=d2(1);
    B(j,5)=d2(1)*hm2^3;
    B(j,6)=3*d2(1)*hm2^2;
    B(j,7)=3*d2(1)*hm2;
    Tm3=(hm2+hm1)^3-hm1^3;
    B(j,8)=(2/3-Tm3*d2(1))/hm1^3;
    B(j,9)=-d2(2)*h2^3;
    B(j,10)=3*d2(2)*h2^2;
    B(j,11)=-3*d2(2)*h2;
    T3=(h2+h1)^3-h1^3;
    B(j,12)=(-2/3-T3*d2(2))/h1^3;
    B(j,16)=d2(2);
    if j==1
        G(1,1)=3*d2(2)*h2^2;
        G(2,1)=-6*d2(2)*h2;
    end
    if j==2
        G(3,1)=3*d2(2)*h2*(h2+2*h1)+3*B(j,12)*h1^2;
        G(4,1)=-6*(d2(2)*h2+B(j,12)*h1);
    end
    if j==3
        G(5,1)=3*d2(1)*hm2^2;
        G(6,1)=6*d2(1)*hm2;
    end
    if j==N-5
        G(1,2)=3*d2(2)*h2^2;
        G(2,2)=-6*d2(2)*h2;
    end
    if j==N-4
        G(3,2)=3*d2(2)*h2*(h2+2*h1)+3*B(j,12)*h1^2;
        G(4,2)=-6*(d2(2)*h2+B(j,12)*h1);
    end
    if j==N-3
        G(5,2)=3*d2(1)*hm2^2;
        G(6,2)=6*d2(1)*hm2;
    end
    
end

function d2=solver(hm2,hm1,h1,h2)
a=h1*hm2*(hm2+hm1)^2;
b=-hm1*h2*(h2+h1)^2;
c=h1^2*hm2*(hm2+hm1)*(hm2+2*hm1);
d=hm1^2*h2*(h2+h1)*(h2+2*h1);
A=[[a b];[c d]];
bb=[2*(hm1+h1)/3; -2*(hm1^2-h1^2)/3];
d2=A\bb;

function g=Bi(x,i,B,Xd)
j=i+1;
if x<Xd(j) || x>Xd(j+4)
    g=0;
elseif x<Xd(j+1)
    g=B(j,4)*(x-Xd(j))^3;
elseif x<Xd(j+2)
    g=B(j,5)+B(j,6)*(x-Xd(j+1))+B(j,7)*(x-Xd(j+1))^2+B(j,8)*(x-Xd(j+1))^3;
elseif x<Xd(j+3)
    g=B(j,9)+B(j,10)*(x-Xd(j+3))+B(j,11)*(x-Xd(j+3))^2+B(j,12)*(x-Xd(j+3))^3;
else
    g=B(j,16)*(x-Xd(j+4))^3;
end


function y=lagrange(x,pointx,pointy)
n=size(pointx,2);
L=ones(n,size(x,2));
if (size(pointx,2)~=size(pointy,2))
   fprintf(1,'\nERROR!\nPOINTX and POINTY must have the same number of elements\n');
   y=NaN;
else
   for i=1:n
      for j=1:n
         if (i~=j)
            L(i,:)=L(i,:).*(x-pointx(j))/(pointx(i)-pointx(j));
         end
      end
   end
   y=0;
   for i=1:n
      y=y+pointy(i)*L(i,:);
   end
end

function final_function=least_squares(f,pointx)
clc;
syms x;

%初始化正交多项式组
p=cell(5,1);

%0次正交多项式
p{1}=1;
%1次正交多项式
p{2}= x;

%2,3,4次正交多项式

for k=2:4
    alpha(k+1) = dot(subs(p{k},x,pointx),subs(p{k}*x,x,pointx))/dot(subs(p{k},x,pointx),subs(p{k},x,pointx));
    beta(k)=dot(subs(p{k},x,pointx),subs(p{k},x,pointx))/dot(subs(p{k-1},x,pointx),subs(p{k-1},x,pointx));
    p{k+1} = (x-alpha(k+1))*p{k}-beta(k)*p{k-1};
end

%得到最终的正交多项式
final_function=0;
for k=1:5
    a(k)=dot(subs(f,x,pointx),subs(p{k},x,pointx))/dot(subs(p{k},x,pointx),subs(p{k},x,pointx));
    final_function=final_function+a(k)*p{k};
end


% end


\end{lstlisting}
\subsection{任务二}
\subsubsection{主函数部分}
\begin{lstlisting}



% Compound trapezoid formula
clc
clear
syms x;
f=@(x) 1./(1+x.^2);
a=-2;
b=2;
n=20;
h=(b-a)/n;
sum=0;
for k=1:1:n-1
  x(k)=a+k*h;
  y(k)=f(x(k));
  sum=sum+y(k);
end
answer=h/2*(f(a)+f(b)+2*sum);
trapz_answer=eval(answer)
ground_truth=2*atan(2)
trapz_bias= abs(trapz_answer-ground_truth)

simpsons_answer = simpsons(f,a,b,n)
simpsons_bias = abs(simpsons_answer-ground_truth)

syms h;
% h=(b-a)/10;
sum=0;
for k=1:1:10-1
  x(k)=a+k*h;
  y(k)=f(x(k));
  sum=sum+y(k);
end
trapz_10_formula=h/2*(f(a)+f(b)+2*sum);
%在trapz_10_formula基础上再次二分之后得到的公式如下


sum=0;
for k=1:1:20-1
  x(k)=a+k*h/2;
  y(k)=f(x(k));
  sum=sum+y(k);
end
trapz_20_formula=h/4*(f(a)+f(b)+2*sum);
%里面的h仍为(b-a)/10


Romberg1_formula = trapz_20_formula*4/3-1/3*trapz_10_formula
Romberg1=eval(subs(Romberg1_formula,h,(b-a)/10))
Romberg1_bias = abs(Romberg1-ground_truth)
%上面的先分10份然后用龙贝格一次加速的任务已完成



syms h;
% h=(b-a)/5;
sum=0;
for k=1:1:5-1
  x(k)=a+k*h;
  y(k)=f(x(k));
  sum=sum+y(k);
end
trapz_5_formula=h/2*(f(a)+f(b)+2*sum);
%上面是分成五份的复合梯形公式。

syms h;
% h=(b-a)/5;
sum=0;
for k=1:1:10-1
  x(k)=a+k*h/2;
  y(k)=f(x(k));
  sum=sum+y(k);
end
trapz_10_formula=h/4*(f(a)+f(b)+2*sum);
%上面是分成十份的复合梯形公式。
%里面的h仍为(b-a)/5

syms h;
% h=(b-a)/5;
sum=0;
for k=1:1:20-1
  x(k)=a+k*h/4;
  y(k)=f(x(k));
  sum=sum+y(k);
end
trapz_20_formula=h/8*(f(a)+f(b)+2*sum);
%上面是分成二十份的复合梯形公式。
%里面的h仍为(b-a)/5

romberg21_formula=trapz_10_formula*4/3-1/3*trapz_5_formula;
romberg22_formula=trapz_20_formula*4/3-1/3*trapz_10_formula;
%上面分别进行了第一次龙贝格加速

romberg2_formula=romberg22_formula*16/15-1/15*romberg21_formula
%这里进行第二次龙贝格加速

romberg2=eval(subs(romberg2_formula,h,(b-a)/5))
romberg2_bias = abs(romberg2-ground_truth)



%下面使用高斯勒让德积分公式
a = linspace(-2,2,11);
%matlab的数组下标从1开始
sum=0;
for i = 1:10
    sum = sum+Gauss_Legendre_quadrature(f,a(i),a(i+1),2);
end
% 复合的 2 点高斯公式
gauss_2points = sum
gauss_2points_bias= abs(gauss_2points-ground_truth)
a = linspace(-2,2,6);
%matlab的数组下标从1开始
sum=0;
for i = 1:5
    sum = sum+Gauss_Legendre_quadrature(f,a(i),a(i+1),4);
end
% 复合的 4 点高斯公式
gauss_4points = sum
gauss_4points_bias= abs(gauss_4points-ground_truth)

\end{lstlisting}
\subsubsection{自定义函数部分}
\begin{lstlisting}

function I = Gauss_Legendre_quadrature(f,a,b,point)
%在-1到1基础上
a0=(b+a)/2;%中心点向右平移的长度
a1=(b-a)/2;%在长度上伸展的倍数

if point==2
    digits(100)
    x0=a0+(a1*(-1/(sqrt(3))));
    x1=a0+(a1*(1/(sqrt(3))));
    fx0=a1*f(x0);
    fx1=a1*f(x1);
    I=fx0+fx1;
end

if point==3
    digits(100)
    x0=a0+(a1*(-vpa(sqrt(3/5))));
    x1=a0;
    x2=a0+(a1*vpa(sqrt(3/5)));
    fx0=a1*vpa(f(x0));
    fx1=a1*vpa(f(x1));
    fx2=a1*vpa(f(x2));
    I=(vpa(5/9)*fx0)+(vpa(8/9)*fx1)+(vpa(5/9)*fx2);
end

if point==4
    digits(100)
    x0=a0+(a1*-vpa((sqrt(3/7-2/7*sqrt(6/5))),100));
    x1=a0+(a1*-vpa((sqrt(3/7+2/7*sqrt(6/5))),100));
    x2=a0+(a1*vpa((sqrt(3/7+2/7*sqrt(6/5))),100));
    x3=a0+(a1*vpa((sqrt(3/7-2/7*sqrt(6/5))),100));
    fx0=a1*f(x0);
    fx1=a1*f(x1);
    fx2=a1*f(x2);
    fx3=a1*f(x3);
    I=((18+vpa(sqrt(30),100))/36*fx0)+((18-vpa(sqrt(30),100))/36*fx1)+((18-vpa(sqrt(30),100))/36*fx2)+((18+vpa(sqrt(30),100))/36*fx3);
end

%Interpretation of results

function I = simpsons(f,a,b,n)
% This function computes the integral "I" via Simpson's rule in the interval [a,b] with n+1 equally spaced points

if numel(f)>1 % If the input provided is a vector
    n=numel(f)-1; h=(b-a)/n;
    I= h/3*(f(1)+2*sum(f(3:2:end-2))+4*sum(f(2:2:end))+f(end));
else % If the input provided is an anonymous function
    h=(b-a)/n; xi=a:h:b;
    I= h/3*(f(xi(1))+2*sum(f(xi(3:2:end-2)))+4*sum(f(xi(2:2:end)))+f(xi(end)));
end



\end{lstlisting}

\subsection{选做题}
\begin{lstlisting}
digits(100)
%下面是选做题:
%先验证复合三点高斯公式收敛阶数
for k=2:8
    sum=0;
    a = linspace(-2,2,2^k+1);
    for i = 1:2^k
        sum = sum+vpa(Gauss_Legendre_quadrature(f,a(i),a(i+1),3),100);
    end
    gauss_3points_bias(k)=abs(sum-ground_truth)
end

for k=2:7
    a = log(gauss_3points_bias(k+1))-log(gauss_3points_bias(k));
    b = log(4/(2^(k+1)))-log(4/(2^(k)));
    order_gauss = a/b
end
%此后的结果趋向于大概是6,说明复合梯形公式余项阶数是6


%验证复合梯形公式收敛阶数
for m=2:6
    % Compound trapezoid formula
    b=2;
    a=-2;
    n=2^m;
    h=(b-a)/n;
    sum=0;
    for k=1:1:n-1
      x(k)=a+k*h;
      y(k)=f(x(k));
      sum=sum+y(k);
    end
    answer=h/2*(f(a)+f(b)+2*sum);
    trapz_answer=eval(answer);
    ground_truth=2*atan(2);
    trapz_bias(m)= abs(trapz_answer-ground_truth);
end
trapz_bias
for k=2:5
    a = log(trapz_bias(k+1))-log(trapz_bias(k));
    b = log(4/(2^(k+1)))-log(4/(2^(k)));
    order_trapz = a/b
end
%此后的结果趋向于大概是2,说明复合梯形公式余项阶数是2


%验证复合辛普森公式收敛阶数
for m=2:6
    a=-2;
    b=2;
    simpsons_answer = simpsons(f,a,b,2^m);
    simpsons_bias(m) = abs(simpsons_answer-ground_truth);
end
simpsons_bias
for k=2:5
    a = log(simpsons_bias(k+1))-log(simpsons_bias(k));
    b = log(4/(2^(k+1)))-log(4/(2^(k)));
    order_simpson = a/b
end
%此后的结果趋向于大概是4,说明复合辛普森公式余项阶数是4
\end{lstlisting}


\end{document}